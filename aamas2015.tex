\documentclass{aamas2015}
\pdfpagewidth=8.5truein
\pdfpageheight=11truein

\begin{document}

\title{Exploiting Object Symmetry for Efficient Grasping}
\numberofauthors{1}

\author{
\alignauthor
Paper XXX
}

\newcommand{\vech}[1]{\textbf{#1}}
\newcommand{\set}[1]{\textbf{#1}}

\maketitle

%% IDEAS
% - force map instead of collision map
% - invariance of height?
% - projection of fingers convex ?
% - add 3rd dimension (saturation)?


\begin{abstract}
In this, paper we introduce an efficient representation for robot grasping that exploits symmetry
properties of objects. The new representation forms a low-dimensional manifold, which can be used to
identify the set of feasible grasps during a sequential manipulation task. We analyse the properties
of this low-dimensional manifold and show that some of these properties can be used for fast
manipulation planning. We apply the introduced representation and planner to bi-manual manipulation
in humanoid robots.
\end{abstract}

% representation (high level / abstract)
% manifold (2 dimensional thing)

\section{Introduction}

\begin{enumerate}
\item What is paper about?
	\begin{enumerate}
	\item Robots are required for assisting in collaborative tasks
	\item Especially physical taks, e.g., domestic environments, healthcare, 
		  defense, manufacturing 
	\item Robots need manipulation and grasping capabilities (one of most 
		  important behaviors)
	\item Grasping is at the heart of planning physical tasks 
	\end{enumerate}

\item What is the problem?
	\begin{enumerate}
	\item Grasp planning taking into account environment and task constraints
	\item Current grasp representations are based on a floating hand, which only 					  represents the end-effector and ignores embodiment of whole robot in
		  environment. As a result, infeasible grasps are generated and have to be
		  pruned out in a post-processing step. 
	\item Well established algorithms assume only one action, and do not include
		  foresight and reasoning about next actions to perform. Example, pouring
		  into a cup. Example, putting mug into washing machine.
	\item Motion planning with multiple subtasks -> planning needs to be fast 
		  as possible (complexity)
	\item !Try to show that separation of planning and grasp selection is naive
	\item Especially in bi-manual, sequential and co-worker scenarios, complexity increases
	\end{enumerate}
	
\item What is our approach?
	\begin{enumerate}
	\item Reoccurring patterns in the geometry of shapes can be exploited
	\item Rotational and linear symmetries and extrusions can be exploited
	\item In this paper we focus on rotational symmetries
	
	\item Object-centered representation
	\item Project world and hand information into the representation, thereby taking
		  into account reachability and collisions
	\item Examples, picture of symm. object with two hands and hands are projected 
		  into manifold.
	\item The symmetric nature of the objects allows us to update our representation
		  as the object rotated around the axis of symmetry.
\end{enumerate}
\item Advantages of approach?
	\begin{enumerate}
	\item We will show that this basic property leads to a significant reduction of 
		  search complexity in grasp planning
	\item \emph{Since the object is symmetric any stable grasp can be rotated around the axis
		  yielding a family of feasible grasps}.
	\item Specifically, this property allows us to generate multiple grasps around 
		  the object which is useful for bimanual and cooperative robot hand over 
		  tasks. In addition, it allows us to plan a single grasp for a sequence of tasks.
	\item We use this approach to generate grasps for a set of scenarios involving
		  sequential, bimanual and collaborative components. 
	\end{enumerate}
	
\item Contributions?
	\begin{enumerate}
	\item new representation for efficient parameterization of stable grasps 	
	\item discussion of properties of representation induced by symmetric objects
	\item a grasp planning algorithms using representation to achieve sequential and 
		  bimanual tasks  
	\end{enumerate}
	\end{enumerate}
	
\section{Related Work}

\section{Manifold Representations for Rotationally Symmetric Objects}
% cyclic manifold 
\begin{enumerate}
	\item Define frames $L$ and $G$ as local object frame and global frame
	\begin{enumerate}
		\item 
		Let $\vech{z}$ be the axis of symmetry, $\vech{o}$ be the origin, and $\vech{a}$ be the polar axis of the object.The polar axis lies in the reference plane of the object and is perpendicular to $\vech{z}$. In the following, we assume that the reference plane is the base of object. Note that $\vech{a}$ can be arbitrarily chosen, since the object is symmetric. Using $\vech{z}$, $\vech{a}$ and their crossproduct we can form a local coordinate frame $L$ for an object.
		
		\item Any point $\vech{p} = [x,y,z]^T$ in the local coordinate frame $L$, can also be represented using a cylindrical parametrization of the coordinate system. 
		
		\item This parametrization leads to a point $[h, r, \theta]^T$ whose components correspond to the height, radius and angle respectively. The height is measured along the axis of symmetry $\vech{z}$, the radius $r$ is the distance between $\vech{p}$ and $\vech{z}$, the angle $\theta$ is the angle between $\vech{a}$ and the projection of $\vech{p} $ onto the local reference plane of the object as can be seen in Fig.~\ref{}. 
		
		\item In the following, we define the radius as a function $r(h)$ of the height, which leads to a two-dimensional parametrization $[h, \theta] \in \mathbb{S}$ of point $\vec{p}$. Subsequently, we can define the function $f: \mathbb{S} \rightarrow \mathbb{R}^3$ that maps the cylindrical coordinates to the 3D local coordinates in the following way:
		
		\begin{equation}
			f(\vech{x}) = f([h, \theta]) = [r(h) ~ cos(\theta), r(h) ~ sin(\theta), h]
		\end{equation}
		
		\item Similarly, we can map from the 3D local coordinate space $L$ to the 
		surface manifold using the inverse mapping, $f^{-1}: \mathbb{R}^3 \rightarrow \mathbb{S}$:
		
		\begin{equation}
			f^{-1}(\vech{p}) = f^{-1}([x,y,z]) = [z, atan2(y,x)]
		\end{equation}
		
		\item Wrap up: Having defined the forward and backward mappings, now we can discuss how this is useful.
	\end{enumerate}
	
	\item Parametrizing Grasps using Low Dimensional Manifolds
	\begin{enumerate}
		
		\item In this section, we demonstrate how to exploit the introduced representation and the symmetry properties of the object to efficiently sample feasible grasps. The key insight is that stable grasps can be rotated around the axis of symmetry without having to modify the hand shape. 
			
		\item We begin the analysis with a simplified grasping model where a point
		on the surface corresponds wrist position during grasping. The hand is assumed to be parallel to the base of the object such that the plane between the grasping fingers is perpendicular to the axis of symmetry. 

		\item The goal is to identify the set of feasible points the robot can grasp. This requires reasoning about reachability and collisions, where we need to ensure that there is a collision-free arm pose with the wrist touching the respective surface point. To sample this space, we propose discretizing the manifold by fixed step sizes for the height $h$ and the angle $\theta$. Refer to picture.
		
		\item For each patch of the discretized manifold, using the center point 
		as the reference for the wrist, we attempt to compute a collision-free 		inverse kinematics solution. Refer to picture again showing red/blue and which collisions were caused. 
		
		
	\end{enumerate}
	
	\item Low-dimensional manifold for grasps on symmetric objects	
	\begin{enumerate}
	\item Let $\vech{x} \in \set{S}, where \vech{x} = [h, \theta]^T$
	\item $h$ is the height of the contact point along the axis of symmetry
	\item $r$
	\end{enumerate}

\end{enumerate}

% name it!
% unwrapping introduces a nonlinear warping
%

% symmetry and invariance
% Pic0a: wiki cylinder parameterization
% Pic0b: discretization and red/blue stuff
% Pic1: example of cup rotating
% Pic2: handprint picture (3 fingers)
% Pic3: multiple hand rotations around object (wine bottle)

\section{Task Planning with Grasp Manifolds}
% Pic4: two tasks superimposed on top of each other (manifold pictures and simulation)
% Pic5: pose of the second task changes (object rotation)
 
\subsection{Planning for Task Sequences}
% Pic6: Rod through 3 tasks
% Pic6.5: Sequence of 3 tasks in simulation

\subsection{Planning for Bi-Manual and Cooperative Tasks}
% Pic7: Hand prints of two robots
% Pic8: Picture of two robots in front of each other, hand over

\section{Experiments}
% offline experiment: how fast is map generation? 
% online experiment: how fast is sequence generation?
% quality? how good? alternative solutions?
% Pic8: 3D blob
% 2 different domains
 
\section{Discussion}
% this representation introduces warping effects
% talk about variable radius objects (picture)

\section{Conclusions}

\end{document}
