\documentclass{aamas2015}
\pdfpagewidth=8.5truein
\pdfpageheight=11truein

\begin{document}

\title{Exploiting Object Symmetry for Efficient Grasping}
\numberofauthors{1}

\author{
\alignauthor
Paper XXX
}


\maketitle

%% IDEAS
% - force map instead of collision map
% - invariance of height?
% - projection of fingers convex ?
% - add 3rd dimension (saturation)?


\begin{abstract}
In this, paper we introduce an efficient representation for robot grasping that exploits symmetry
properties of objects. The new representation forms a low-dimensional manifold, which can be used to
identify the set of feasible grasps during a sequential manipulation task. We analyse the properties
of this low-dimensional manifold and show that some of these properties can be used for fast
manipulation planning. We apply the introduced representation and planner to bi-manual manipulation
in humanoid robots.
\end{abstract}

% representation (high level / abstract)
% manifold (2 dimensional thing)

\section{Introduction}

\begin{enumerate}
\item What is paper about?
	\begin{enumerate}
	\item Robots are required for assisting in collaborative tasks
	\item Especially physical taks, e.g., domestic environments, healthcare, 
		  defense, manufacturing 
	\item Robots need manipulation and grasping capabilities (one of most 
		  important behaviors)
	\item Grasping is at the heart of planning physical tasks 
	\end{enumerate}

\item What is the problem?
	\begin{enumerate}
	\item Grasp planning taking into account environment and task constraints
	\item Current grasp representations are based on a floating hand, which only 					  represents the end-effector and ignores embodiment of whole robot in
		  environment. As a result, infeasible grasps are generated and have to be
		  pruned out in a post-processing step. 
	\item Well established algorithms assume only one action, and do not include
		  foresight and reasoning about next actions to perform. Example, pouring
		  into a cup. Example, putting mug into washing machine.
	\item Motion planning with multiple subtasks -> planning needs to be fast 
		  as possible (complexity)
	\item !Try to show that separation of planning and grasp selection is naive
	\item Especially in bi-manual, sequential and co-worker scenarios, complexity increases
	\end{enumerate}
	
\item What is our approach?
	\begin{enumerate}
	\item Reoccurring patterns in the geometry of shapes can be exploited
	\item Rotational and linear symmetries and extrusions can be exploited
	\item In this paper we focus on rotational symmetries
	
	\item Object-centered representation
	\item Project world and hand information into the representation, thereby taking
		  into account reachability and collisions
	\item Examples, picture of symm. object with two hands and hands are projected 
		  into manifold.
	\item The symmetric nature of the objects allows us to update our representation
		  as the object rotated around the axis of symmetry.
\end{enumerate}
\item Advantages of approach?
	\begin{enumerate}
	\item We will show that this basic property leads to a significant reduction of 
		  search complexity in grasp planning
	\item \emph{Since the object is symmetric any stable grasp can be rotated around the axis
		  yielding a family of feasible grasps}.
	\item Specifically, this property allows us to generate multiple grasps around 
		  the object which is useful for bimanual and cooperative robot hand over 
		  tasks. In addition, it allows us to plan a single grasp for a sequence of tasks.
	\item We use this approach to generate grasps for a set of scenarios involving
		  sequential, bimanual and collaborative components. 
	\end{enumerate}
	
\item Contributions?
	\begin{enumerate}
	\item new representation for efficient parameterization of stable grasps 	
	\item discussion of properties of representation induced by symmetric objects
	\item a grasp planning algorithms using representation to achieve sequential and 
		  bimanual tasks  
	\end{enumerate}
	\end{enumerate}
	
\section{Related Work}

\section{Manifold Representations for Rotationally Symmetric Objects}
% name it!
% symmetry and invariance
% Pic1: example of cup rotating
% Pic2: handprint picture (3 fingers)
% Pic3: multiple hand rotations around object (wine bottle)

\section{Task Planning with Grasp Manifolds}
% Pic4: two tasks superimposed on top of each other (manifold pictures and simulation)
% Pic5: pose of the second task changes (object rotation)
 
\subsection{Planning for Task Sequences}
% Pic6: Rod through 3 tasks
% Pic6.5: Sequence of 3 tasks in simulation

\subsection{Planning for Bi-Manual and Cooperative Tasks}
% Pic7: Hand prints of two robots
% Pic8: Picture of two robots in front of each other, hand over

\section{Experiments}
% offline experiment: how fast is map generation? 
% online experiment: how fast is sequence generation?
% quality? how good? alternative solutions?
% Pic8: 3D blob
% 2 different domains
 
\section{Discussion}

\section{Conclusions}

\end{document}
